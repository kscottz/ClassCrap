\documentclass{article}

\usepackage{fullpage} % Include this if you want to cram lots of things on a page
 
\usepackage{amsmath} % these are standard macro packages of the American Mathematical Society
\usepackage{amssymb}
\usepackage{hyperref}
%\usepackage{stmaryrd}

\usepackage{epsfig} % if you want figures

%\usepackage{fancyhdr} % These 4 lines are needed to set up the running  header
%\fancyhead[LE,RO]{Katherine Scott - Homework 1}
%\fancyhead[RE,LO]{\thepage}
%\pagestyle{fancy}

\newcommand{\matlab}[1]
{\centerline{\parbox{.9\textwidth}{\noindent\textsc{\bf MATLAB:} #1}}}

\newcommand{\code}[1]{\texttt{#1}}

\newcommand {\x}{\V{x}}
\newcommand {\y}{\V{y}}
\newcommand {\V}[1]{\mbox{\boldmath$#1$}}
% To add some paragraph space between lines.
% This also tells LaTeX to preferably break a page on one of these gaps
% if there is a needed pagebreak nearby.
\newcommand{\blankline}{\quad\pagebreak[2]}

\begin{document}
\title{Computer Vision: Homework 4}

\author{Katherine A. Scott}
\maketitle
\mbox{}
\begin{center}
\href{mailto:katherineAScott@gmail.com}{kas2221@columbia.edu}

\end{center}
\section{Problem 1}
\subsection{Part A}
\[
img(i,j)=\alpha^{\prime\prime} +2\alpha^{\prime} + A(c+k)+Bf 
\]
This means the total number of calculations (ignoring the calculations need to start a new row) is $4MN$ additions and $3MN$ additions. For large images the setup cost is negligible and these terms dominate. 
\section{Problem 2}
Given the Hough equation 
\[
0 = x\sin{\theta}-y\cos{\theta}+\rho
\]
we rearrange to get:
\[
\rho = y\cos{\theta} - x\sin{\theta}
\]
Now we want to get this in the form of $y=A\cos{(\theta+\phi)}$ so we can rationalize about the underlying sinusoid. We convert $x$ and $y$ to polar coordinates, namely $x=r\cos{\phi}$ and $y=r\sin{\phi}$ to get:
\[
\rho = r \cos{\theta}\sin{\phi}-r\sin{\theta}\cos{\phi}
\]
We now use the following trig identity:
\[
\sin{(\theta-\beta)}=\sin{\theta}\cos{\beta}-\cos{\theta}\sin{\beta}
\]
Which yields (replacing r with its true value):
\[
\rho = \sqrt{x^{2}+y^{2}}
\sin{(\phi-\theta)}
\]
Where
\[
\phi = cos^{-1}{(\frac{x}{\sqrt{x^{2}+y^{2}}})}
\]
It is clear from this equation that in Hough space $x$ and $y$ directly relate to the phase and amplitude of the resulting sinusoid, with the length of the point from the origin relating to the amplitude and the angle from y-axis relating to the phase shift. The period/frequency of the sinusoid is completely independent of the $x$ and $y$ values.
\end{document}