\documentclass{article}

\usepackage{fullpage} % Include this if you want to cram lots of things on a page
 
\usepackage{amsmath} % these are standard macro packages of the American Mathematical Society
\usepackage{amssymb}
\usepackage{hyperref}
%\usepackage{stmaryrd}

\usepackage{epsfig} % if you want figures

%\usepackage{fancyhdr} % These 4 lines are needed to set up the running  header
%\fancyhead[LE,RO]{Katherine Scott - Homework 1}
%\fancyhead[RE,LO]{\thepage}
%\pagestyle{fancy}

\newcommand{\matlab}[1]
{\centerline{\parbox{.9\textwidth}{\noindent\textsc{\bf MATLAB:} #1}}}

\newcommand{\code}[1]{\texttt{#1}}

\newcommand {\x}{\V{x}}
\newcommand {\y}{\V{y}}
\newcommand {\V}[1]{\mbox{\boldmath$#1$}}
% To add some paragraph space between lines.
% This also tells LaTeX to preferably break a page on one of these gaps
% if there is a needed pagebreak nearby.
\newcommand{\blankline}{\quad\pagebreak[2]}

\begin{document}
\title{Computer Vision: Homework 5}

\author{Katherine A. Scott}
\maketitle
\mbox{}
\begin{center}
\href{mailto:katherineAScott@gmail.com}{kas2221@columbia.edu}

\end{center}
\section{Problem 1}
The general formula for the motion of our point is
\[
P(t)=P_0+Vt= \begin{pmatrix} x_0 \\ y_0 \\ z_0 \end{pmatrix} + \begin{pmatrix} u \\ v \\ w \end{pmatrix}t
\]

We can push this through our pin hole camera model to get the
following:
\[
x_i(t)=f(\frac{x_0}{z_0}+t\frac{u}{w})
\]
\[
y_i(t)=f(\frac{y_0}{z_0}+t\frac{v}{w})
\]
Which shows that our x and y position as function of time fits a nice
linear equation of the form $y=mx+b$. The superposition of these
should also be linear, giving us a line in image space. 


\section{Problem 2}

\end{document}